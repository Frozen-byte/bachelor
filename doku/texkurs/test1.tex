\documentclass[paper=a4,12pt]{scrreprt}
\usepackage[ngerman]{babel}
\usepackage[utf8x]{inputenc}
\usepackage[round,authoryear]{natbib}
\usepackage[hyphens]{url}
\usepackage{amssymb}
\usepackage{hyperref}
\usepackage[]{graphicx}
\usepackage[all]{hypcap}
\usepackage{eurosym}
\title{Natzn\\Projektdokumentation}
\author{Lennart Przygode, Malte Dethlefs, Julian Roosch}
\hypersetup{
    colorlinks,
    citecolor=black,
    filecolor=black,
    linkcolor=black,
    urlcolor=black
}
\begin{document}
\maketitle
\tableofcontents
\begin{abstract}
In diesem Projekt haben wir uns mit der Entwicklung eines Frameworks zur Implementation von Online-Spielen auseinandergesetzt - der Fokus lag dabei auf rundenbasierten Gesellschaftsspielen.
\end{abstract}

\section{Bedienungsanleitung}
\section{Entwicklerhandbuch}
\euro 
\section{Softwarearchitektur}
\begin{displaymath}
\left.\frac{e^{j2\pi ft}-e^{-j2\pi ft}}{2j}\right|_{t=-T_0}^{T_0}
\end{displaymath}
\section{Projektauswertung}
\subsection{Was funktionierte?}
\subsection{Was lief unerwartet?}
\subsection{Ausblick}
\subsection{Tatsächlicher Aufwand}
\end{document}
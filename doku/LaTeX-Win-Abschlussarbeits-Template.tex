%%%%%%%%%%%%%%%%%%%%%%%%%%%%%%%%%%%%%%%%%%%%%%%%%
%------ LaTeX-Template f�r Abschlussarbeiten, Prof. Thomas G�rne, Dezember 2012 --------
%%%%%%%%%%%%%%%%%%%%%%%%%%%%%%%%%%%%%%%%%%%%%%%%%

%---- Header (mit Formateinstellugen) laden, Inputencoding pr�fen ------

\input{hawmt-abschlussarbeits-header}

%\usepackage[applemac]{inputenc} % Inputencoding f�r Mac
%\usepackage[latin1]{inputenc} % Inputencoding f�r PC/Win
%\usepackage[utf8]{inputenc} % Inputencoding, universell
\usepackage[utf8x]{inputenc} % Inputencoding, universell


%------------------------ Titelblatt-Layout laden ----------------------------------

\input{hawmt-bachelor-titelblatt}
%\input{hawmt-master-titelblatt}

%---------------------------- Titeldefinitionen --------------------------------------

\newcommand{\vorname}{Patrick}
\newcommand{\nachname}{Hilgenstock}
\newcommand{\matrikelnummer}{2203656}

\newcommand{\titel}{Oregon\\[0.2ex] 
				\Large Eine Team-basierte Web-Applikation zur Stoff-Vertiefung}

\newcommand{\erstpruef}{Prof. Dr. Edmund Weitz}
\newcommand{\zweitpruef}{Prof. Dr Torsten Edeler}

%\date{vorl�ufige Fassung vom \today}   % praktisch f�r Vorab-Versionen. 
\date{\sffamily Hamburg, 20.02.2017}  % Abgabedatum!

%--------------------------------------------------------------------------------------
%----------------------------- hier gehts los! --------------------------------------
%--------------------------------------------------------------------------------------

\begin{document}
\selectlanguage{ngerman}
\maketitle           % Titelseite erzeugen
\tableofcontents % Inhaltsverzeichnis erzeugen
\clearpage          % Seitenumbruch


\chapter{Einleitung}

\section{Zielsetzung}

Das Ziel dieser Arbeit ist es eine Webapplikation zu erstellen, welche bei der Gestaltung einer Mathematik-Vorlesung helfen soll. Dabei werden von der Anwendung Aufgaben generiert und an den Nutzer gesendet. Dieser kann diese nun lösen und seine Antwort abschicken. Nach der Validierung bekommt er ein Feedback und bei dem korrekten Lösen eine neue Aufgabe, sowie einige Punkte für sein Team. Der Administrator kann dabei den ganzen Vorgang überwachen und den aktuellen Punktestand einsehen. \\
Einer der Kernpunkte der Anwendung wird es sein, dass dynamisch Aufgaben generiert werden. Das bedeutet, dass wenn der Nutzer mehrere Aufgaben anfordert er jedes mal eine Aufgabe bekommt welche zwar aus dem selben Themenbereich kommen aber doch jedes mal unterschiedliche Variablen beinhalten und so ein erneutes Berechnen erfordern.


\section{Bereits existierende Software}

Wenn man sich die bereits existierenden Lösungen für dieses Problem anschaut stößt man auf sehr viele Programme die sich mit dem Vertiefen von Stoff befassen. Allerdings setzen diese lediglich darauf den Nutzer Aufgaben zu stellen und die Aufgaben zu validieren. Es gibt keinerlei Möglichkeit die vorhandenen Aufgaben zu erweitern oder die Ergebnisse mit anderen Leuten zu teilen, geschweige denn automatisch eine Übersicht über die Ergebnisse eines ganzen Kurses zu erschaffen. Ebenfalls als Nachteil aufzufassen ist hier die Tatsache, dass es sich bei fast allen Lösungen um Programme handelt welche käuflich zu erwerben sind\\

Als Variante zur Validierung der Ergebnisse eines ganzen Kurses besteht die Webanwendung "MARS" (Minimal Audience Response System). Dieses ist sehr gut dafür geeignet statische Fragen zu stellen und zu validieren. Allerdings ist es hier nicht möglich dynamisch Aufgaben zu erstellen. Zusätzlich kann nur ein einziges Mal eine Antwort abgegeben werden, es ist also nicht möglich mehrere Fragen des selben Bereiches zu beantworten und so sein Wissen intensiver zu testen. \\

Ansonsten gibt es bereits viele Webseiten die sich mit dem Stellen von Aufgaben aus verschiedenen Themenbereichen kümmern, zum Beispiel RegexGolf ( \url{https://alf.nu/RegexGolf} ). Hierbei handelt es sich um eine Anwendung bei der dem Nutzer eine Liste von Wörtern gegeben wird. Die Aufgabe besteht nun darin einen regulären Ausdruck zu finden welcher für einen Teil der Liste matcht und bei der anderen nicht.

\section{Vorteile der neuen Lösung}

Bei der neuen Anwendung soll es sich um ein Projekt handelt welches alle Vorteile existierender Programme vereint, während es die Nachteile auslässt. Das bedeutet, dass die Anwendung viel Wert darauf legt erweiterbar zu sein, ein möglichst großes Spektrum an Aufgaben abdecken zu können und dynamisch Aufgaben zu generieren können. \\
Gleichzeitig wird es sich bei der resultierenden Webapplikation um Open Source handeln, das heißt der Quellcode ist frei verfügbar und kostenlos, ebenso wie die Webapplikation selbst.

\chapter{Konzeption der Anwendung}

\section{Analyse des Anwendungsfeldes}

Bevor man sich nun an die Entwicklung der Anwendung macht bleibt eine Frage offen. Für wen ist sie eigentlich gemacht und wie soll sie eingesetzt werden? \\
Als grobe Eingrenzung lässt sich sagen, dass diese Anwendung für die Lehre geschaffen wird, um genau zu sein für den Teilbereich der Mathematik. Wichtig ist hierbei zu beachten, dass sich die Applikation nach dem momentanen Konzept nur während der Vorlesung / des Unterrichts benutzen lässt. Dies liegt daran, dass die Idee Teams vorsieht, welche zusammenarbeiten und so Punkte sammeln. Die Arbeit als einzelne Person ist nicht vorgesehen. \\
Da die Anwendung vorsieht, dass jeder Nutzer einen eigenen Zugang zum Internet hat verlagert sich das Hauptnutzungsfeld in höhere Bildungsstufen, zum Beispiel Universität und Fachhochschule, da dort jeder ein eigenes Smartphone oder sogar einen Laptop benutzt. \\
Zusammenfassend lässt sich also sagen, dass die Anwendung hauptsächlich im Bereich der Lehre der Mathematik in Universitäten und Fachhochschulen zum Einsatz kommen wird.

\section{Was ist eine Aufgabe}

Als letzten Schritt ist nun noch zu definieren was genau eine Aufgabe eigentlich ist, beziehungsweise wie sie im Sinne dieser Anwendung gesehen wird. \\
Für diese Arbeit werden wir Aufgaben immer in drei Teilen betrachten:
\begin{enumerate}
\item Die Variablen \\
Als erstes betrachten wir die Variablen. Sie werden für jede Aufgabe neu generiert. Dadurch, dass diese dynamisch generiert wird jedes mal wenn eine Aufgabe angefragt wird.
\item Die Darstellung \\
An zweiter Stelle befassen wir uns mit der Darstellung der Aufgabe. Um die vorher generierten Variablen für den menschlichen Nutzer angemessen darzustellen wird hier eine Zeichenkette generiert, welche die Aufgabe repräsentiert, in die dann die Variablen eingesetzt werden.
\item Die Lösung \\
Nachdem die Aufgaben generiert und dargestellt wurden bleibt noch eine Sache übrig. Die Validierung der Aufgabe. Es wird die Lösung des Nutzers übergeben und festgestellt ob diese korrekt gelöst wurde oder nicht.
\end{enumerate}


\section{Analyse der Anforderungen}

\section{Verwendete Technologien}

\subsection{Spring-Boot}

\subsection{Angular 2}




\chapter{Ein Rundgang durch das Frontend}





\section{Die Ansicht des Nutzers}


\subsection{Die Auswahl des Teams}

\subsection{Erhalten und bearbeiten der Aufgaben}



\section{Die Sicht des Administrators}
\subsection{Bearbeitung der Aufgaben-Generatoren und ihren Hilfsmitteln}
\subsection{Starten eines neuen Aufgaben-Generators}
\subsection{Übersicht über die laufende und letzte Aufgabe}
\subsection{Der Fehler-Log}




\chapter{Ein Blick unter die Haube - das Backend}

\section{Die Sicherung der REST-API}

\section{Dokumentation der REST-Schnittstellen}

\section{Die Generierung und Validierung der Aufgaben}

\section{Die Sicherung der Aufgaben-Generierung}



\chapter{Fazit}




\section{Das Ergebnis}


\section{Wie kann die Anwendung verbessert / erweitert werden}



\begin{thebibliography}{}

% Formatierung f�r Internetquelle
% Grundregel: Name, Vorname (falls vorhanden), V�-Jahr (falls vorhanden), Titel in Anf�hrungszeichen, URL, Datum des letzten Aufrufs
% zur Formatierung der URL unbedingt den url-Befehl benutzen!!!
\bibitem[Blu-ray Disc Association(2005)]{bluray} 
Blu-ray Disc Association: 
\emph{White paper Blu-ray Disc Format 2.B Audio Visual Application, Format Specifications for BD-ROM}, 
\url{http://www.blu-raydisc.com/Assets/downloadablefile/2b_bdrom_audiovisualapplication_0305-12955-15269.pdf}, 2005, letzter Zugriff: 1. 10. 2012

% Formatierung f�r Aufsatz / Paper: Titel in Anf�hrungszeichen, Zeitschriftentitel kursiv
\bibitem[Dooley \& Streicher(1982)]{dooley_streicher} 
Dooley, Wesley L.  \& Streicher, Ronald D.:
\glqq M--S Stereo: A Powerful Technique for Working in Stereo\grqq, 
\emph{Journ. Audio Engineering Society} vol. 30 (10), 1982

% Formatierung f�r Fachbuch, Diplomarbeit o.�.: Titel kursiv
\bibitem[Kuttruff(1991)]{kuttruff}
Kuttruff, Heinrich: 
\emph{Room Acoustics}, 3. Aufl., Elsevier 1991

% Formatierung f�r Fachbuch mit Herausgeber und mehreren Autoren
\bibitem[Spehr(2009)]{spehr}
Spehr, Georg (Hrsg.): 
\emph{Funktionale Klnge}, transcript 2009

% Formatierung f�r ein einzelnes Kapitel eines speziellen Autors aus einem Fachbuch mit mehreren Autoren
\bibitem[Sowodniok(2009)]{sowodniok}
Sowodniok, Ulrike: 
\glqq Funktionaler Stimmklang -- Ein Prozess mit Nachhalligkeit\grqq, 
in: Spehr, Georg (Hrsg.): \emph{Funktionale Klnge}, transcript 2009

% Formatierung f�r Aufsatz / Paper: Titel in Anf�hrungszeichen, Zeitschriftentitel kursiv
\bibitem[Stephenson(1990)]{stephenson}
Stephenson, Uwe: 
\glqq Comparison of the Mirror Image Source Method and the Sound Particle Simulation Method\grqq, 
\emph{Applied Acoustics} vol. 29, 1990


\end{thebibliography}

%--------------------- EIGENST�NDIGKEITSERKL�RUNG ---------------
\clearpage\thispagestyle{empty}
\eigen  % im header definiert
%--------------------------------------- ENDE ------------------------------------
\end{document}
%%%%%%%%%%%%%%%%%%%%%%%%%%%%%%%%%%%%












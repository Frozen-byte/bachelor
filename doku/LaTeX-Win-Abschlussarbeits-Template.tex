%%%%%%%%%%%%%%%%%%%%%%%%%%%%%%%%%%%%%%%%%%%%%%%%%
%------ LaTeX-Template f�r Abschlussarbeiten, Prof. Thomas G�rne, Dezember 2012 --------
%%%%%%%%%%%%%%%%%%%%%%%%%%%%%%%%%%%%%%%%%%%%%%%%%

%---- Header (mit Formateinstellugen) laden, Inputencoding pr�fen ------

\input{hawmt-abschlussarbeits-header}

%\usepackage[applemac]{inputenc} % Inputencoding f�r Mac
%\usepackage[latin1]{inputenc} % Inputencoding f�r PC/Win
%\usepackage[utf8]{inputenc} % Inputencoding, universell
\usepackage[utf8x]{inputenc} % Inputencoding, universell


%------------------------ Titelblatt-Layout laden ----------------------------------

\input{hawmt-bachelor-titelblatt}
%\input{hawmt-master-titelblatt}

%---------------------------- Titeldefinitionen --------------------------------------

\newcommand{\vorname}{Patrick}
\newcommand{\nachname}{Hilgenstock}
\newcommand{\matrikelnummer}{2203656}

\newcommand{\titel}{Oregon\\[0.2ex] 
				\Large Eine Team-basierte Web-Applikation zur Stoff-Vertiefung}

\newcommand{\erstpruef}{Prof. Dr. Edmund Weitz}
\newcommand{\zweitpruef}{Prof. Dr Torsten Edeler}

%\date{vorl�ufige Fassung vom \today}   % praktisch f�r Vorab-Versionen. 
\date{\sffamily Hamburg, 20.02.2017}  % Abgabedatum!

%--------------------------------------------------------------------------------------
%----------------------------- hier gehts los! --------------------------------------
%--------------------------------------------------------------------------------------

\begin{document}
\selectlanguage{ngerman}
\maketitle           % Titelseite erzeugen
\tableofcontents % Inhaltsverzeichnis erzeugen
\clearpage          % Seitenumbruch


\chapter{Einleitung}

\section{Zielsetzung}

Das Ziel dieser Arbeit ist es eine Webapplikation zu erstellen, welche bei der Gestaltung einer Mathematik-Vorlesung helfen soll. Dabei werden von der Anwendung Aufgaben generiert und an den Nutzer gesendet. Dieser kann diese nun lösen und seine Antwort abschicken. Nach der Validierung bekommt er ein Feedback und bei dem korrekten Lösen eine neue Aufgabe, sowie einige Punkte für sein Team. Der Administrator kann dabei den ganzen Vorgang überwachen und den aktuellen Punktestand einsehen. \\
Einer der Kernpunkte der Anwendung wird es sein, dass dynamisch Aufgaben generiert werden. Das bedeutet, dass wenn der Nutzer mehrere Aufgaben anfordert er jedes mal eine Aufgabe bekommt welche zwar aus dem selben Themenbereich kommen aber doch jedes mal unterschiedliche Variablen beinhalten und so ein erneutes Berechnen erfordern.


\section{Bereits existierende Software}

Wenn man sich die bereits existierenden Lösungen für dieses Problem anschaut stößt man auf sehr viele Programme die sich mit dem Vertiefen von Stoff befassen. Allerdings setzen diese lediglich darauf den Nutzer Aufgaben zu stellen und die Aufgaben zu validieren. Es gibt keinerlei Möglichkeit die vorhandenen Aufgaben zu erweitern oder die Ergebnisse mit anderen Leuten zu teilen, geschweige denn automatisch eine Übersicht über die Ergebnisse eines ganzen Kurses zu erschaffen \\

Als Variante zur Validierung der Ergebnisse eines ganzen Kurses besteht die Webanwendung "MARS" (Minimal Audience Response System). Dieses ist sehr gut dafür geeignet statische Fragen zu stellen und zu validieren. Allerdings ist es hier nicht möglich dynamisch Aufgaben zu erstellen. Zusätzlich kann nur ein einziges Mal eine Antwort abgegeben werden, es ist also nicht möglich mehrere Fragen des selben Bereiches zu beantworten und so sein Wissen intensiver zu testen. \\

Ansonsten gibt es bereits viele Webseiten die sich mit dem Stellen von Aufgaben aus verschiedenen Themenbereichen kümmern, zum Beispiel RegexGolf ( \url{https://alf.nu/RegexGolf} ). Hierbei handelt es sich um eine Anwendung bei der dem Nutzer eine Liste von Wörtern gegeben wird. Die Aufgabe besteht nun darin einen regulären Ausdruck zu finden welcher für einen Teil der Liste matcht und bei der anderen nicht.

\section{Vorteile der neuen Lösung}

Die Vorteile der neuen Lösung bestehen hauptsächlich darin, die Vorteile bereits bestehender Lösungen zu kombinieren, während die Nachteile ausgemerzt werden. \\

Zusätzlich wird es sich hierbei um ein Open Source Projekt handeln, das heißt, dass das Projekt ebenfalls jederzeit eingeesehen werden kann um daraus zu lernen oder eigene Schlüsse zu ziehen. Gleichzeitig bedeutet Open Source auch frei verfügbar, jeder der diese Applikation benötigt und die notwendige Hardware zur Verfügung hat wird jederzeit in der Lage sein einen Server mit dem Projekt aufzusetzen und dieses für die eigenen Zwecke zu nutzen \\

Nicht zu vergessen ist auch der Aspekt der Erweiterbarkeit. Sollten sich nach Abschluss der ersten Iteration neue Anforderungen ergeben so ist man in der Lage diese jederzeit in das bestehende Projekt zu integrieren und es zu erweitern. \\

Doch am wichtigsten ist der Aspekt, der dynamischen Aufgabengeneriung, auf welchen besonders viel Wert gelegt wird. Nur so kann sichergestellt werden, dass ein Wiederverwertungswert entsteht und das Wissen eines Nutzers in der Tiefe getestet werden kann.

\section{Was ist eigentlich eine Aufgabe}

Bevor mit der eigentlichen Arbeit angefangen werden konnte stand am Anfang eine Frage: Was genau ist eigentlich eine Aufgabe? In welche Teile kann man sie aufteilen und wie kann man eine Aufgabe beschreiben? \\
Um vernünftig mit einer Aufgabe umgehen zu können wird sie in Zukunft wie folgt definiert : \\
Bei einer Aufgabe handelt es sich um Konstrukt welches in drei Teile aufteilbar ist. \\
1. Die Variablen \\
Jede Aufgabe benötigt einen bestimmten Satz an Variablen die nach bestimmten Regeln gewählt werden können. Das ändern der Variablen ändert nichts an dem Sinn oder der Art der Aufgabe.\\
2. Die Darstellung \\
Jede Aufgabe besitzt eine Darstellungsart über die sie an einen menschlichen Nutzer übermittelt werden, der dadurch in der Lage ist die Aufgabe zu verstehen und zu bearbeiten. \\
3. Die Lösung \\



\chapter{Ein Blick unter die Haube - Das Backend}

\section{Die Wahl des Frameworks}

Aufzählung vorhandener Frameworks und deren Vor / Nachteile
Warum Spring Boot

\section{Die Datenbankanbindung}

\section{Die Sicherung des Backends}


\section{Die Endpunkte der Schnittstelle}


\chapter{Die Sicht des Anwenders - Das Frontend}

\section{Die Wahl des Frameworks}

Durch die Definition der Vorteile der neuen Anwendung werden auch direkt Anforderungen an das Framework mit dem gearbeitet wird gestellt. Das Wichtigste ist die klare Abtrennung der verschiedenen Abschnitte sowie der Erweiterbarkeit. Wenn man sich auf dem aktuellen Markt umschaut kommt ein klarer Kanditat dafür in Frage: Angular 2. \\
Aufzählung Vorteile Angular sowie integrierte Patterns

\section{Die Sicherung auf der obersten Ebene}

Login-Komponente
Speichern JWT-Token
Anhängen von Token an jeden Request

\section{Die Sicht des Nutzers}

\subsection{Die Auswahl des Teams}

\subsection{Das Bearbeiten der Aufgaben}

\section{Die Sicht des Administrators}

\subsection{Übersicht über die laufende Runde}

\subsection{Das Starten einer neuen Runde}

\subsection{Verwaltung der Generatoren}



\chapter{Fazit}

\section{Das Ergebnis}

\section{Wie kann die Anwendung verbessert / erweitert werden}


\begin{thebibliography}{}

% Formatierung f�r Internetquelle
% Grundregel: Name, Vorname (falls vorhanden), V�-Jahr (falls vorhanden), Titel in Anf�hrungszeichen, URL, Datum des letzten Aufrufs
% zur Formatierung der URL unbedingt den url-Befehl benutzen!!!
\bibitem[Blu-ray Disc Association(2005)]{bluray} 
Blu-ray Disc Association: 
\emph{White paper Blu-ray Disc Format 2.B Audio Visual Application, Format Specifications for BD-ROM}, 
\url{http://www.blu-raydisc.com/Assets/downloadablefile/2b_bdrom_audiovisualapplication_0305-12955-15269.pdf}, 2005, letzter Zugriff: 1. 10. 2012

% Formatierung f�r Aufsatz / Paper: Titel in Anf�hrungszeichen, Zeitschriftentitel kursiv
\bibitem[Dooley \& Streicher(1982)]{dooley_streicher} 
Dooley, Wesley L.  \& Streicher, Ronald D.:
\glqq M--S Stereo: A Powerful Technique for Working in Stereo\grqq, 
\emph{Journ. Audio Engineering Society} vol. 30 (10), 1982

% Formatierung f�r Fachbuch, Diplomarbeit o.�.: Titel kursiv
\bibitem[Kuttruff(1991)]{kuttruff}
Kuttruff, Heinrich: 
\emph{Room Acoustics}, 3. Aufl., Elsevier 1991

% Formatierung f�r Fachbuch mit Herausgeber und mehreren Autoren
\bibitem[Spehr(2009)]{spehr}
Spehr, Georg (Hrsg.): 
\emph{Funktionale Kl�nge}, transcript 2009

% Formatierung f�r ein einzelnes Kapitel eines speziellen Autors aus einem Fachbuch mit mehreren Autoren
\bibitem[Sowodniok(2009)]{sowodniok}
Sowodniok, Ulrike: 
\glqq Funktionaler Stimmklang -- Ein Prozess mit Nachhalligkeit\grqq, 
in: Spehr, Georg (Hrsg.): \emph{Funktionale Kl�nge}, transcript 2009

% Formatierung f�r Aufsatz / Paper: Titel in Anf�hrungszeichen, Zeitschriftentitel kursiv
\bibitem[Stephenson(1990)]{stephenson}
Stephenson, Uwe: 
\glqq Comparison of the Mirror Image Source Method and the Sound Particle Simulation Method\grqq, 
\emph{Applied Acoustics} vol. 29, 1990


\end{thebibliography}

%--------------------- EIGENST�NDIGKEITSERKL�RUNG ---------------
\clearpage\thispagestyle{empty}
\eigen  % im header definiert
%--------------------------------------- ENDE ------------------------------------
\end{document}
%%%%%%%%%%%%%%%%%%%%%%%%%%%%%%%%%%%%











